% article example for classicthesis.sty
\documentclass[10pt,a4paper]{article} % KOMA-Script article scrartcl
\usepackage{lipsum}
\usepackage{url}
\usepackage[nochapters]{classicthesis} % nochapters

\usepackage[utf8x]{inputenc} % codifica scrittura

\usepackage[T1]{fontenc}
\usepackage[square,numbers]{natbib}
\usepackage{amsmath, amsthm, amssymb, amsfonts, tikz}
\usetikzlibrary{snakes}
\usepackage{titletoc}
\usepackage{verbatim}
\usepackage{hyperref}
\usepackage{boxedminipage}

\begin{document}
    \title{\rmfamily\normalfont\spacedallcaps{AI exam exercise}}
    \author{\spacedlowsmallcaps{massimo nocentini - 5422207}}
    \date{\today}

    \maketitle

    \begin{abstract}
        This short document explains my work to support
        AI exam, taught by Prof. Paolo Frasconi at University of Florence,
        comments some results and shows how to reproduce test cases.
        I implemented a Prolog program that produce a schedule for the
        problem at hand, satisfying the requested constraints.
    \end{abstract}

%    \tableofcontents

    \section{Problem Analysis}
    In this exercise we tackle a scheduling problem on classes allocation
    for a generic university. It is a small instance of a real scenario
    and in the following sections we describe the considered constraints,
    Prolog predicates that fullfil them and how to describe a scheduling
    dataset by predicates ready for computation.
    We can formalize the problem as follow:
    \begin{itemize}
        \item a set of rooms $R = \lbrace r_1, \ldots, r_n \rbrace$, each
            one of them is described with a name and the number
            of available sittings;
        \item a set of teachings $T = \lbrace t_1, \ldots, t_m \rbrace$, each
            one of them is described with a name and the hours taught in a week;
        \item a set of professors $P = \lbrace p_1, \ldots, p_s \rbrace$, each
            one of them is described with a name and a subset of $T$ of teachings
            it is capable to teach;
        \item a set of degree courses $C = \lbrace c_1, \ldots, c_q \rbrace$, each
            one of them is described with a name, a subset $T^\prime$ of teachings
            and a number of students interested in each offered teaching;
        \item a subset of courses $C^\prime$ could share a subset $T^\prime$ of teachings.
    \end{itemize}

    \subsection{Implemented Constraints}
    Over the previous problem definition it is required to produce a
    schedule, ie a set of ``teaching slots'' over a standard set of
    working days $D = \lbrace Monday, Thusday, Wednesday, Thursday, Friday \rbrace$,
    such that the following constraints hold:
    \begin{itemize}
        \item given a day $d \in D$, no time overlapping for each professor $p$,
            ie if $p$ teach two different teachings then their time slots
            have to be disjoint;
        \item given a day $d \in D$, no time overlapping for each room $r$,
            ie if in $r$ two different teachings happen to be taught then
            their time slots have to be disjoint;
        \item to each professor $p$ at most $2$ teachings can be assigned
            (while it is possible to design a dataset where $p$ is capable
            to teach more teachings);
        \item given a day $d \in D$, each professor $p$ cannot teach more than
           $4$ hours;
        \item given a day $d \in D$, if a teaching $t$ is taught in $d$ then
            $t$ have to be taught for at least $2$ and at most $4$ hours;
        \item if a teaching $t$, shared by a subset of courses
            $\lbrace (c_1, t, s_1), \ldots, (c_n, t, s_n)\rbrace$,
            is taught in a room $r = (name, s)$ then $s_1 + \ldots + s_n \leq s$,
            where $s$ is the number of available sittings in $r$ and
            $s_i$ is the number of students interested in teaching $t$ enrolled
            to course $c_i$.
    \end{itemize}

    \subsection{Prolog Predicates}
    In order to produce a schedule such that all constraints are met, I
    implemented a Prolog program which consist of a main file \footnote{
       find it in \emph{clp-fd/scheduler.prolog}} and in this section I describe
    quickly the most important predicates and the used coding technique, while
    in the next section it is described how to describe a problem instance.

    A top level predicate \textbf{go/1} is introduced as an interface in order
    to solve the supplied scheduling problem, printing the running time and a
    solution which, for a simple instance, looks like as the following:
    \begin{verbatim}
[schedule(monday,time(8,10),auditorium,computer_science,bianchi,[engineering]),
schedule(tuesday,time(8,10),auditorium,computer_science,bianchi,[engineering]),
...
schedule(monday,time(12,14),auditorium,algebra,verdi,[engineering,medicine])]
    \end{verbatim}
    Let $ts$ be a ground term representing a ``teaching slot''. If $ts$ belongs to a
    solution, such that $ts = schedule(d, time(s,e), r, t, p, cs)$,
    then $ts$ has the following meaning:
    \begin{quote}
        on day $d$, professor $p$ teaches teaching $t$ in room $r$,
        starting at $s$ till $e$, and every student interested in following $t$
        which is enrolled to course $c \in cs$ can attend the class
    \end{quote}

    Variables about day, professor, teaching, room and timings
    are instantiated within \textbf{valid/6} predicate.
    Every time a variable is instantiated some
    checks are performed in order to fail as soon as possible if such
    instantiation is not a good one. This shows the ``accumulator''
    coding technique and it explains why predicate \textbf{valid/6} has
    such an high ariety, since collecting arguments needs to be passed around.
    An entry predicate \textbf{valid/1} is defined in order to build
    initial values for such collecting parameters.
    Observe that using this coding style, has been
    possible to code \textbf{valid/6} as a \emph{tail recursive} predicate.

    Some auxiliary predicates have been defined in order to take apart
    the complexity in small pieces, each one ensuring a requested constraint,
    to name a few of them: \textbf{ensure\_professor\_teaches\_at\_most\_two\_teachings/4},
    \textbf{ensure\_no\_overlapping\_for\_teaching\_in\_the\_same\_day/5} and so on.


    \subsection{Specifying datasets}
    An instance of a problem can be defined using the following predicates:
    \begin{itemize}
        \item \textbf{room($n$, $s$)} means $n$ is the name
            of a room containing $s$ sittings;
        \item \textbf{teaching($t$, $h$)} means $t$ is the name
            of a teaching which is taught $h$ hours a week;
        \item \textbf{teaches($p$, $t$)} means $p$ is the name
            of a professor which can teach teaching $t$;
        \item \textbf{course($c$, $t$, $s$)} means $s$ students
            enrolled in course $c$ are interested in following
            teaching $t$, offered by that course.
    \end{itemize}
    In order to express that a teaching $t$ is shared by a subset of
    courses $\lbrace c_1, \ldots, c_n \rbrace$ write down $n$ clauses
    as follows: $ course(c_1, t, s_1), \ldots, course(c_n, t, s_n)$.

    Moreover, if two courses $c_1, c_2$ both should offer a teaching $t$ about
    a given topic but their programs do not allow to share, then $t$
    have to be splitted in two teachings $t_1, t_2$ to be associated as follow :
    $course(c_1, t_1, s_1), course(c_2, t_2, s_2)$,
    assigning professors (not necessary different) to $t_1, t_2$.

    \section{Test cases}
    In this section it is reported a table of test cases with a
    quick description and the metodology used to run and reproduce them.


    \subsection{Experiments table}
    In \ref{table:test-cases} a test case is represented by a row, where
    the first column is the \emph{make rule} to be used in order to rerun the
    test case, while the second column is a brief description about what
    aspect the test points to.

    \begin{table}
        \label{table:test-cases}
        \begin{tabular}{ c | p{6cm} }
            make rule & Description  \\
            \hline
            \textbf{no-overlapping-for-professor} & A single professor $p$
                teaches two teachings and two rooms are available.
                This test shows that no teaching time overlapping happen for $p$ \\
            \hline
            \textbf{no-overlapping-for-room} & A room $r$ is available and two
                professors teach a different teaching respectively.
                This test shows that no teaching time overlapping happen for $r$  \\
            \hline
            \textbf{more-than-two-teaching-single-professor} & A room $r$ is available and
                there exists only one professor $p$. Moreover $p$ have to teach three teachings.
                This test shows that this problem has no solution and it takes
                a long time to fail since Prolog attempts every possible paths. \\
            \hline
            \textbf{more-than-two-teaching-multi-professors} & A room $r$ is available and
                there are one more professor $p^\prime$ respect the test above,
                which is capable to teach a course also taught by $p$.
                As before $p$ can teach three teachings.
                This test shows that this problem is indeed satisfiable but
                the order in which teaching clauses are written implies different
                running times \\
            \hline
            \textbf{at-most-four-day-hours-for-prof} & A room $r$ is available and
                there exists only one professor $p$
                which is capable to teach two courses for a comprehensive 18 hours
                a week. This test shows that $p$ teaches at most $4$ hours a day\\
            \hline
            \textbf{teaching-with-one-hour-a-week} & A room $r$ is available and
                there exists only one professor $p$ and a teaching $t$ such that
                $p$ teaches $t$, with the setting that $t$ has only $1$ hour a week.
                This test shows that this little instance is not satisfiable,
                since if a teaching is scheduled on a day $d$, it should be taught
                for at least two hours \\
            \hline
            \textbf{teaching-shared-by-courses-too-many-students} &
                A room $r$ is available,
                there exists only one professor $p$ and a teaching $t$ taught by $p$
                and shared by two courses $c_1, c_2$ such that the total number
                of interested students is greater than the number of sittings in
                room $r$.
                This test shows that this little instance is not satisfiable \\
            \hline
            \textbf{teaching-shared-by-courses-no-overlapping} &
                This instance is quite similar as the above one, a new
                room $r^\prime$ is available, there exists one more professor
                $p^\prime$ that also can teach $t$, which is shared by two
                courses $c_1, c_2$ as before. Here the total number
                of interested students is less than the number of sittings in
                both rooms. This test shows that no overlapping for teaching $t$
                happens, nonetheless auxiliary room and professor are available \\
            \end{tabular}

        \caption{Test cases table with make rules and descriptions}
    \end{table}

    \subsection{How to reproduce experiments}
    All the work to produce this document has been versioned in a
    \emph{Git} repository hosted at:
    \url{https://github.com/massimo-nocentini/ai-exercise}\\
    In order to reproduce the above test cases just checkout the
    repository, change directory into \emph{clp-fd} and use \textbf{make}
    unix facility, specifying as a rule one in the first column of table
    \ref{table:test-cases}, for example:
    \begin{verbatim}
$ git clone https://github.com/massimo-nocentini/ai-exercise.git
... git clone output ...
$ cd ai-exercise/clp-fd/
$ make no-overlapping-for-professor
... a solution to a simple dataset ...
    \end{verbatim}
    A make rule reports an error if and only if the corresponding
    test case fails if and only if there's no solution for the
    problem instance under test.

\end{document}
